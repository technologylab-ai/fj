\documentclass[10pt]{article}
\usepackage[utf8]{inputenc}
\usepackage[T1]{fontenc}
\usepackage[ngerman]{babel} % For German language rules
\usepackage{geometry}
\geometry{a4paper, margin=1in} % Set page margins
\usepackage{enumitem} % For custom list formatting (to control indents)
\usepackage{xurl} % For better line breaks in URLs/paths (like long filenames)

\usepackage{config-defaults}
% Load (and override defaults)
\IfFileExists{config.tex}{%
  \input{config.tex}%
}{%
  % Silent fallback if file doesn't exist — nothing to do
}

% Remove page numbers for a clean look
\pagestyle{empty}

% Disable paragraph indentation globally
\setlength{\parindent}{0pt}
\setlength{\parskip}{1em} % Add space between paragraphs

\begin{document}

\section*{\centering Reiseprotokoll \FjCompanyName} % Centered heading
\subsection*{\centering Reisender: [Ihr vollständiger Name, z.B. Max Mustermann]} % Centered subheading

\noindent\rule{\textwidth}{0.4pt} % Horizontal line

\vspace{10pt} % Add some vertical space below the line

\textbf{Reiseziel:}
Name der Stadt/Region, Land, z.B. \textit{"Wien, Österreich"} oder \textit{"München, Deutschland"}

\vspace{1em} % Space between sections

\textbf{Reisezeitraum:}
\begin{itemize}[nosep, leftmargin=1.5em, itemsep=0.2em] % Custom leftmargin for clean look
    \item \textbf{Beginn:} TT.MM.JJJJ, HH:MM Uhr, z.B. \textit{01.07.2025, 08:00 Uhr}
    \item \textbf{Ende:} TT.MM.JJJJ, HH:MM Uhr, z.B. \textit{03.07.2025, 17:30 Uhr}
\end{itemize}

\vspace{1em}

\textbf{Reisezweck:}
Beschreiben Sie hier den genauen, betrieblichen Anlass der Reise. \textbf{Das ist entscheidend für die steuerliche Anerkennung!}
\begin{itemize}[nosep, leftmargin=1.5em, itemsep=0.2em]
    \item Kundengespräch bei Mustermann GmbH bezüglich Projekt X zur Einführung neuer Softwarelösungen
    \item Teilnahme an der Fachmesse 'Digital Solutions 2025' zur Marktanalyse und Netzwerkpflege mit potenziellen Partnern
    \item Weiterbildung 'SEO für Fortgeschrittene' bei der Academy GmbH zur Verbesserung der Online-Sichtbarkeit der eigenen GmbH
    \item Besuch des Lieferanten ABC in XYZ zur Qualitätssicherung und Verhandlung neuer Lieferkonditionen
\end{itemize}

\vspace{1em}

\textbf{Transportmittel:}
\begin{itemize}[nosep, leftmargin=1.5em, itemsep=0.2em]
    \item z.B. \textit{"ÖBB Bahn"}, \textit{"Lufthansa Flug"}, \textit{"Privat-PKW"}, \textit{"Mietwagen"}
    \item \textbf{Bei Nutzung Privat-PKW (für Kilometergeld):}
    \begin{itemize}[nosep, leftmargin=2em, itemsep=0.2em] % Slightly deeper indent for sub-list
        \item Startadresse: Ihre Startadresse, z.B. \textit{Musterstr. 1, 1010 Wien}
        \item Zieladresse(n): Genaue Adresse(n) des Geschäftspartners/Veranstaltungsortes, z.B. \textit{Konferenzzentrum, Messeplatz 1, 8010 Graz}
        \item Gefahrene Kilometer (gesamt Hin- \& Rückfahrt): Zahl km, z.B. \textit{400 km}
    \end{itemize}
\end{itemize}

\vspace{1em}

\textbf{Übernachtung:}
\begin{itemize}[nosep, leftmargin=1.5em, itemsep=0.2em]
    \item z.B. \textit{"Hotel Musterblick, Wien"} oder \textit{"Keine Übernachtung erforderlich"} (wenn Sie am selben Tag zurückfahren)
\end{itemize}

\vspace{1em}

\textbf{Liste der zugehörigen Belege (hochgeladen in BMD.COM, im Ordner "Eingangsrechnungen"):}
\small % Make this section slightly smaller to save space and look cleaner
Bitte hier die \textbf{Dateinamen} der gescannten Belege auflisten, die zu dieser spezifischen Reise gehören. Achten Sie auf die einheitliche Benennung mit Präfix (z.B. Datum\_Ziel\_Beschreibung)!
\begin{itemize}[nosep, leftmargin=1.5em, itemsep=0.2em]
    \item \texttt{JJJJ-MM-TT\_Reiseziel\_Reiseprotokoll.pdf} (dieses Dokument)
    \item \texttt{JJJJ-MM-TT\_Reiseziel\_Hotel\_Musterblick.pdf} \textit{(Beispiel: \texttt{2025-07-01\_Wien\_Hotel\_Musterblick.pdf})}
    \item \texttt{JJJJ-MM-TT\_Reiseziel\_OEBB\_Ticket.pdf}
    \item \texttt{JJJJ-MM-TT\_Reiseziel\_Taxibeleg\_Flughafen.pdf}
    \item \texttt{JJJJ-MM-TT\_Reiseziel\_Geschaeftsessen\_Restaurant\_X.pdf} (nur, falls Sie dieses Essen nicht über die Tagesgeldpauschale abrechnen, sondern separat belegen wollen)
    \item \texttt{JJJJ-MM-TT\_Reiseziel\_Parkgebuehr\_Bahnhof.pdf}
\end{itemize}

\normalsize % Reset font size

\vspace{1em}

\textbf{Anmerkungen (optional):}
Hier können Sie zusätzliche Informationen festhalten, die für die Buchhaltung relevant sein könnten, z.B.:
\begin{itemize}[nosep, leftmargin=1.5em, itemsep=0.2em]
    \item Mittagessen am \textit{02.07.2025} wurde vom Kunden bezahlt. (Wichtig, da dies die Tagesgeldpauschale kürzen kann)
    \item Hotelkosten wurden bereits direkt über Firmenkreditkarte bezahlt.
    \item Grund für höhere Kosten: Kurzfristige Buchung aufgrund dringendem Termin.
    \item Beleg für Fahrt von Hotel zu Meeting vom \textit{02.07.2025} (Taxi) fehlt leider, da Quittung verloren.
\end{itemize}

\noindent\rule{\textwidth}{0.4pt} % Horizontal line

\end{document}
